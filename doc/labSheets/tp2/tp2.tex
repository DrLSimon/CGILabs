\documentclass{article}
\usepackage[linktocpage=true]{hyperref}
\usepackage{graphicx}
\usepackage[T1]{fontenc} 
\usepackage[utf8]{inputenc}
\usepackage{subfigure}
\usepackage{color}
\def\todo#1{}%{{\color{red} #1}}
\def\fixme#1{}%{{\color{green} #1}}

\author{Loïc Simon}
\title{BSI TP2: Application de textures}
\date{\today}

\begin{document}
\maketitle

% Objectif<<<
L'objectif de ce TP est de mettre en place les techniques basiques de placage de
texture, à savoir
\begin{itemize}
\item le chargement d'une texture dans le GPU (à partir d'un fichier TGA),
\item les directives de coloration dans le fragment shader
\item la définition de coordonnées UV (dans le VAO),
\item le filtrage (magnification et minification),
\item le mipmapping
\end{itemize}
%>>>

\section{Exercices}
{\bf NB:} Pour l'exercice 1, vous devrez modifier le fichier
\verb|utils/textures.cpp|, et pour les suivant le fichier
\verb|Textures/textures.cpp|. Il s'agit bien de deux fichiers différents. Vous
aurez aussi à modifier le fragment shader \verb|texture.f.glsl|.         

\subsection{Exercice 1: Chargement d'une texture}%<<<
Dans le fichier \verb|utils/textures.cpp| remplir le code de la fonction \verb|loadTGATexture| pour effectuer le
chargement d'un fichier de texture dans le CPU puis la stocker dans le GPU.
Vous utiliserez notamment les routines \verb|read_tga| pour charger la texture
dans la mémoire CPU puis \verb|glTexImage2D|, pour la stocker dans le GPU.

Utiliser la routine \verb|glTexParameteri| pour contrôler les paramètres de
wrapping (extension de la texture en dehors de son domaine naturel $[0,1]$) et
de filtrage.

Dans cet exercice, vous aurez aussi besoin d'éditer le fichier
\verb|texture.f.glsl| pour mettre en place la coloration par texture dans le
fragment shader. Vous devrez utiliser la routine GLSL \verb|texture|.

Lorsque ces deux étapes seront réalisées, la scène sera uniformément texturée
par un damier. Vous devrez alors éditer le fichier \verb|Textures/textures.cpp|
dans la fonction \verb|draw()| pour associer une texture de dé a certains objets
de la scène. Pour cela vous devrez associer la bonne texture a l'objet uniform
de te type sampler2D du programme de shader utilisé pour texturer.

\begin{itemize}
  \item[Q1.] Expliquez succinctement et avec vos propres mots le rôles des
  paramètres de la fonction \verb|glTexImage2D|.
  \item[Q2.] Essayez plusieurs paramètres de wrapping (extrapolation) et de
  filtrage (interpolation) différents,
  et expliquez leurs effets. Illustrez par des captures d'écran.
\end{itemize}
%>>>

\subsection{Exercice 2: Coordonnées uv}%<<<

Ouvrez le fichier \verb|dice_texture.tga|, et observez le. Remplacez les valeurs
du buffer \verb|correct_uv_buffer_data| pour que chacun des numéros se plaque
correctement sur une face du cube. Vous pourrez aussi télécharger d'autres
textures qui vous plaisent et les plaquer sur le cube. Pour cela vous les
convertirez en \verb|tga| en utilisant la commande suivante:

\begin{verbatim}
 convert input.jpg -type TrueColor output.tga 
\end{verbatim}

\begin{itemize}
  \item[Q3.] Commentez sur la difficulté pour obtenir un mapping cohérent pour
  différentes textures.
\end{itemize}
%>>>

\subsection{Exercice 3 et 4: Mip-mapping et filtrage anisotrope}%<<<

Dans la fonction \verb|drawPlane|, gênerez des mipmaps et activez les. Ensuite,
mettez en place un filtrage anisotrope.

\begin{itemize}
  \item[Q4.] Expliquez la différence entre la magnification et la minification
  (dans le contexte du placage de textures).
  \item[Q5.] Expliquez quels défauts visuels le mipmapping cherche a éliminer.
  Expliquez le principe de fonctionnement de cette technique, et notamment
  pourquoi elle permet d'éliminer les effets indésirables susmentionnés.
  \item[Q6.] De même, expliquez le rôle et le principe du filtrage anisotrope.
\end{itemize}
%>>>

\subsection{Exercice 5: Fragment shader}%<<<

Changez la définition de fragColor, pour implementer des rendus originaux
(par exemple animés).
Vous pourrez utiliser la texture, la couleur, et toutes autres entrées (\verb|in| ou
\verb|uniform|) que vous aurez passées au shader.


\begin{itemize}
  \item[Q7.] Décrivez ce que vous avez entrepris ici.
\end{itemize}
%>>>

\end{document}

% vim: fileencoding=utf-8 spelllang=fr:
